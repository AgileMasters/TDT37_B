\documentclass{article}
\usepackage[utf8]{inputenc}
\usepackage[norsk, english]{babel}
\usepackage{cite}
\usepackage{hyperref}
\usepackage{color}
\usepackage{rotating}
\usepackage{adjustbox}

\begin{document}

\title{Views on the Agile Retrospective}
\date{\today{}}
\author{Bjørn Dølvik}

\maketitle

\paragraph{Introduction}
lol
The agile retrospective is a process used within agile development to bind people together and open opportunities for team to improve on their current development practices. It is a practice that are conducted in several different ways, by many different people. Through this article I will look at the agile retrospective through different perspectives and connect this to existing literature. I'll be focusing on three main topics: Participation, knowledge and group awareness. But first I'll go a little bit deeper into the agile retrospective.

\paragraph{The agile retrospective}
As previously mentioned the agile retrospective is a process used within agile development. Usually it takes place in the end of a development iteration, often called sprint within scrum. The retrospective are performed in different ways, but there are some common techniques used as KJ-sessions, root cause analysis and fishbone diagrams. KJ-sessions are done by separating all participants, make them write opinions on the last iteration of the project, and when all are done they get together and each participant then explains their comment to the rest of the group. root cause analysis can then be performed on these comments were all the participants group the different comments together and discusses their causes. Fishbone diagrams may also be applied for this reason were the participants takes each big cause and find subcauses for each cause until they are agree that they found the root problems. When the retrospective activities are finished the group find a set of the problems identified, which they will try to work on for the next iteration. Through these activities development teams are able to improve on they development practices. 

\paragraph{}
The participants in an agile retrospective varies from team to team. Some teams include the customer, others include management and someone only includes the team of developers. Including different participants depends on what kinds on practices one wanna improve. If one want to become better at including customers in the project development it could be a benefit to include them in the retrospective as well. However if one want to improve on the internal practices of a development team it might be wise not to include the customers. By including different kinds of participants one might get different results from performing an agile retrospective. 

\paragraph{}
Agile retrospectives is a practice that supports sharing of knowledge. Through the agile retrospective participants are required to share their experiences through verbal communication, thus making tacit knowledge explicit. This can later be recorded for sharing with other teams or in a company. This may also support communities of practice as participants gets experience in converting tacit knowledge into explicit. By using performing agile retrospective one trains participants in knowledge conversion. 

\paragraph{}
Through agile retrospectives group awareness is increases as a result of practice. As all the participants are presenting in KJ-sessions and contributing in cause analysis it raises awareness to the others participating. The other participants may then acquire these experiences and such become aware of what others are thinking of the current practices used. Thus it increases the awareness as everyone shares their opinions on the current state of the project.
 
\paragraph{Existing literature}

\end{document}